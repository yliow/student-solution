%-*-latex-*-
\sectionthree{Algorithmic analysis: how fast is an algorithm?}
\begin{python0}
from solutions import *; clear()
\end{python0}

Some text.
Some text.
Some text.
Some text.
Some text.
Some text.
Some text.
Some text.
Some text.
Some text.
Some text.

The following is an example:

\input{exercises/runtime-of-sum-of-squares/main.tex}

Some text.
Some text.
Some text.
Some text.
Some text.
Some text.
Some text.
Some text.
Some text.
Some text.
Some text.

To include your exercise, start by placing it in the
\verb!exercises! directory. For example, if your exercise
directory is named \verb!tree-6!, you should see the
following output when running \verb!ls! inside the
\verb!exercises! directory:

\begin{Verbatim}[frame=single, fontsize=\small]
runtime-of-sum-of-squares  tree-6
\end{Verbatim}

Next, to display the exercise in this document, from the main directory open:
\[
\verb!algorithm-analysis-how-fast-is-an-algorithm.tex!
\]
then locate the placeholder:
\[
\verb!%???%!
\]
and replace it with:

\[
\verb!\input{exercises/tree-6/main.tex}!
\]

%???%

Now, go back to the main directory and run the \verb!make! command.
This should compile the document with your exercise included.

To add your answer, navigate to the directory of your exercise and
enter your solution in the file named \verb!answer.tex!. Return to
the main directory, run \verb!make! again, and your solution should
appear on the following pages of this document.

